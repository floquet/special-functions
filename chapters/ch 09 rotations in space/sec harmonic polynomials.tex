\section[Harmonic polynomials and representations of $O(3)^{+}$]{Harmonic polynomials and \\\qquad \ representations of $O(3)^{+}$}

Another possible method of constructing representations of the rotation group is to consider homogeneous polynomials of fixed degree in the variables $x$, $y$, and $z$. The methods described in section \S \eqref{sec:8-2} can be applied to construct representations in the invariant subspaces of such functions.	This method, which was used successfully in section \S \eqref{sec:8-2}, is less satisfactory for the present problem since it does not generate irreducible representations. This method will, however, be discussed in this section to demonstrate the relation between the group representations obtained in the previous section and the important functions, the spherical harmonics.

We consider the space $S_{l}$ of homogeneous polynomials of degree $l$ in the variables $x$, $y$, $z$. This space is spanned by the monomials of the form $x^{m}y^{n}z^{l-m-n}$. There are
\begin{equation*}
  \sum_{m=0}^{l} \sum_{m=0}^{l{\rd{ -m }}} =
  \sum_{m=0}^{l} \paren{l - m + 1} =
  \frac{\paren{l + 2} \paren{l + 1}} {2}
\end{equation*}
such monomials, which are clearly linearly independent, so that $S_{l}$ is of dimension $(l + 2)(l + 1)/2$. The representations defined by $S_{l}$ must be reducible since $S_{l}$ contains an invariant subspace, the space $S_{l-2}$ of all polynomials of the form $(x^{2}+y^{2}+z^{2}) P_{l-2}$ where $P_{l-2}$ is a polynomial of degree $l-2$. Since the group is orthogonal, the function $x^{2}+ y^{2} + z^{2}$ is invariant under the group transformations and $S_{l-2}$ is an invariant subspace (of dimension $l( l - 1)/ 2$).	The representation defined by $S_{l}$ can be assumed to be unitary, in which case the subspace $S_{l-2}^{\perp}$ orthogonal to $S_{l-2}$ is also invariant. This subspace is of dimension
\begin{equation*}
  \frac{(l + 2)(l + 1)} {2} - \frac{l(l - 1)}{2} = 2l + 1
\end{equation*}
This invariant subspace is rather nebulous since no inner product has been defined on $S_{l}$. It is Possible, however, to construct a $(2l + 1)-$dimensional invariant subspace of $S_{l}$ in another way. We consider the mapping of $S_{l}$ onto $S_{l-2}$ defined by
\begin{equation*}
  P_{l}\paren{x} \rightarrow  P_{l-2}\paren{x} = \nabla^{2} P_{l}\paren{x}
\end{equation*}
where the image of $P_{l}$ is obviously in $S_{l-2}$. It is convenient to denote the points whose coordinates are in $\paren{x,y,z}$ by $\textbf{x}$. It will now be shown that, for any rotation $\\R$,
\begin{equation*}
  \nabla^{2} \paren{P_{l}\paren{\R^{-1}x}} = P_{l-2}\paren{\R^{-1}x}
\end{equation*}
where $P_{l-2}\paren{x} = \nabla^{2} P_{l}\paren{x}$. We will denote $\R^{-1}\x$ by $\x'$. If the elements of $\R$ are $a_{ij}$ the component $j$ of $\x'$ is given by $\sum_{i} a_{ij} 
x_{i}$. We can now write
\begin{equation*}
  \begin{split}
    \sum_{i} \pdt{}{\x_{i}^{2}} P_{l}\paren{\R^{-1} \x}  &=
    \sum_{ijk} \pdt{P_{l}} {\x_{j}' \x_{k}'} \paren{\x'} \pd{\x_{j}'}{\x_{i}} \pd{\x_{k}'}{\x_{i}} \\
    &=\sum_{ijk} a_{ij} a_{ik} \pdt{P_{l}}{\x_{j}' \x_{k}'} \paren{\x'} \\ 
    &=\sum_{j} \pdt{P_{l}}{{\rd{ \x_{j}'^{2} }}} \paren{\x'} \\ 
    &= P_{l-2} \paren{\R^{-1}\x} 
  \end{split}
\end{equation*}
The implication of this result is that the mapping $\nabla^{2}$ from $S_{l}$ onto $S_{l-2}$ satisfies
\begin{equation}
  \nabla^{2} D^{l}\paren{\R} = D^{l-2}\paren{\R}\nabla^{2}
\label{9-25}
\end{equation}
where $D^{l}\paren{\R}$ are defined by equation \eqref{6.42}.

We consider now the subspace $H_{l}$ of $S_{l}$ composed of polynomials $P_{l}$ satisfying
\begin{equation}
  \nabla^{2} P_{l} = 0.
\label{9-26}
\end{equation}
A function satisfying this equation, which is Laplace's equation, is said to be \textbf{harmonic}. It follows immediately from \eqref{9-25} that $H_{l}$ is invariant; if $\nabla^{2} P_{l} = 0$,
\begin{equation*}
  \nabla^{2} D^{l} \paren{\R}  P_{l} = D^{l-2} \paren{\R} \nabla^{2} P_{l} = 0
\end{equation*}
and $D^{l}\paren{\R}P_{l}$ is also harmonic. It will now be shown that the representation of the rotation group generated by $H_{l}$, the set of harmonic polynomials of degree $l$, is equivalent to the representation $D^{l}\paren{\R}$ defined by \eqref{9-19}.

It is possible to obtain $2l + 1$ linearly independent solutions of \eqref{9-26} explicitly. For this purpose it is convenient to introduce new variables 
\begin{equation*}
  \begin{split}
    u &= \half \paren{x + i y}, \\
    v &= \half \paren{x - i y} 
  \end{split}
\end{equation*}
in terms of which \eqref{9-26} becomes
\begin{equation}
  \pdt{P_{l}} {u \partial v} + \pdt{P_{l}} {z^{2}} = 0.
\label{9-27}
\end{equation}
If $P_{l}$ is a homogeneous polynomial of degree $l$ in $u$, $v$, and $z$ it is also a homogeneous polynomial of degree $l$ in $x$, $y$, and $z$. It is possible to write down four solutions of \eqref{9-27}, $u^{l}$, $v^{l}$, $u^{l-1}z$, $v^{l-1}z$ immediately. More generally, we look for a solution that contains a term of the form $u^{l-m_{v}m}$, $m = 0, 1, \dots l$. This is not a solution
since
\begin{equation}
  \nabla^{2} u^{l-m}v^{m} = \paren{l-m}m u^{l-m-1}v^{m-1} .
\end{equation}
It is possible to eliminate the right-hand side by adding to $u^{l-m}v^{m}$ a term $(-1)(l-m)mu^{l-m-1}v^{m-1}z^{2}/2$. One then obtains
  % % % EQUATION
  \begin{multline*}
    \nabla^{2}\paren{u^{l-m}v^{m} - \frac{(l-m)m u^{l-m-1}v^{m-1}z^{2}} {2}} \\
    = - \frac{(l-m)(l-m-1)m(m-1)u^{l-m-2}v^{m-2}z^{2}} {2}
  \end{multline*}
  % %
It is now possible to add a third term, $(l-m)(l - m - 1)m(m - 1) u^{l-m-2}v^{m-2}/4!$ to eliminate the new term on the right-hand side. Proceeding in this way one eventually obtains a harmonic polynomial of degree $l$ which can be written
  % % % EQUATION
  \begin{equation}
    f_{lm}(u,v,z) = \sum_{p}(-1)^{p} \frac{(l-m)!m!} {(l-m-p)!(m-p)!(2p)!} u^{l-m-p} v^{m-p} z^{2p}
  \label{9-28}
  \end{equation}
  % %
The sum on $p$ is from 0 to the smaller of $m$ and $l-m$. It can be verified by direct substitution into \eqref{9-27} that $f_{lm}$ is a harmonic polynomial. The functions $f_{lm}$ have the further property, which will prove to be important, that the difference of the exponents of $u$ and $v$,
$l-2m$, is the same for each term. We note that there are $(l + 1)$ functions $f_{lm}$.

In a similar way, it is possible to find solutions that contain a term $u^{l-m}v^{m-1}z$, where $m = 1,2, \dots, l$. These solutions can be written
  % % % EQUATION
  \begin{multline}
    g_{lm} = \sum_{p}(-1)^{p} \frac{(l-m)!(m-1)!}{(l-m-p)!(m-p-1)!(2p+1)!} u^{l-m-p} v^{m-p-1} z^{2p+1}
  \label{9-29}
  \end{multline}
  % %
In this case the index $p$ runs from 0 to the smaller of $m- 1$ and $l- m$. There are $l$ such solutions with the property that the difference of the exponents of $u$ and $v$ is $l-2m + 1$ for each term in the sum.

There are altogether $(2l + 1)$ functions $f_{lm}$, $g_{lm}$. These can be labeled by an index $s$, 
$$s = -l, -l + 1, ..\dots, l- 1, l,$$
$s$ being the difference of the exponents of $u$ and $v$ in each term of a particular function. These functions will be denoted by $k_{ls}$. The functions $k_{ls}$ are obviously linearly independent since no two of them can contain the same monomial. It can also be seen that the functions $k_{ls}$ span $H_{l}$. Let $P_{l}(x)$ be a harmonic polynomial of degree $l$. Consider a term $u^{p-m}v^{q-m}z^{l-p-q+2m}$ in $P_{l}$; it can be seen from \eqref{9-27} that the coefficient of every term in $P_{l}$ of the form $u^{p}v^{q}z^{l-p-q}$ is uniquely determined by the coefficient of $u^{p}v^{q}z^{l-p-q}$. In fact, all the terms of this form must occur as a constant multiple of $k_{l,p-q}$, and can be removed by subtracting $ck_{l,p-q}$ from $P_{l}$ for some $c$. It is, therefore, apparent that $P_{l}$ can be expressed as a linear combination of the $k_{ls}$.

The functions $k_{ls}$ generate, by equation \eqref{eq:6-44}, a representation of the proper rotation group. This representation will be denoted by $\Delta^{l}(R)$. We will not calculate $\Delta^{l}(R)$ explicitly but rather show that it is equivalent to the representation $D^{l}(R)$ defined by \eqref{9-19}. It will be shown first that $\Delta^{l}(R)$ is diagonal if $R$ is a rotation by an angle $\phi$ about the $z$ axis. Under the inverse of such a rotation, the spatial variables are transformed according to
\begin{equation*}
  \begin{split}
    x & \to \phantom{-}x\cos \phi + y\sin {\rd{\phi}}, \\
    y & \to -x\sin \phi + y \cos \phi, \\
    z & \to z .
  \end{split}
\end{equation*}
It follows that $u$ is transformed according to 
$$u\to (x\cos \phi + y\sin \phi) + i(-x\sin \phi + y \cos \phi) = e^{-i\phi}x + i e^{-i\phi}y = e^{-i\phi}u.$$ 
Similarly, $v = u^{*}$ is transformed to $e^{i\phi} v$. It follows that a monomial of the form $u^{\alpha}v^{\beta}z^{\gamma}$ is transformed to $e^{i(\beta-\alpha){\rd{\phi}}}u^{\alpha}v^{\beta}z^{\gamma}$, and hence that
%
$$k_{ls}\paren{R^{-1}x} = c^{-is \phi}k_{ls}(x)$$
%
since each term of $k_{ls}$ is multiplied by the same factor $e^{is\phi}$ From \eqref{6.44} we can write, for $R$ a rotation by $\phi$ about the $z$ axis,
  % % % EQUATION
  \begin{equation}
    \Delta^{l}(R)_{st} = e^{is\phi}\delta_{st}
  \label{9-30}
  \end{equation}
  % % %
Each class of the rotation group has been shown to contain a rotation about the $z$ axis. Comparison of \eqref{9-23a} and \eqref{9-30} shows that rotations about the $z$ axis are represented by the same matrices in $D$ and $\Delta$; the characters of the two representations are, therefore, the same and the representations are equivalent.

We consider now the matrix $M$ that transforms $D^{l}(R)$ to $\Delta^{l}(R)$:
  % % % EQUATION
  \begin{equation}
    M^{-1} \Delta^{l}(R)M = D^{l(R)}
  \label{9-31}
  \end{equation}
  % % %
for all $R$. If $R$ is, in particular, a rotation about the $z$ axis, $\Delta^{l}(R) = D^{l}(R)$ and $M$ satisfies $M D^{l}(R) = D^{l}(R) M$. In this case $D^{l}(R)$ is, however, diagonal with diagonal elements which are in general different. It has been seen previously that this implies $M$ is diagonal; the matrix elements of $M$ will, therefore, be denoted by $\mu_{i}\delta_{ij}$. Equation \eqref{9-31} can now be written
  % % % EQUATION
  \begin{equation}
    \Delta^{l}(R)_{mn} = \mu_{m} D^{l}(R)_{mn} \mu_{n}^{-1} .
  \label{9-32}
  \end{equation}
  % % %
The functions $k_{ls}(x)$ and the representations $\Delta^{l}(R)$ are related by
  % % % EQUATION
  \begin{equation*}
    k_{lt}\paren{R^{-1}x} = \sum_{s}\Delta^{l}(R)_{st}k_{ls}(x) .
  \end{equation*}
  % % %
Substituting \eqref{9-32} into this relation yields the result
  % % % EQUATION
  \begin{equation}
    \mu_{t} k_{lt} \paren{R^{-1}x}\sum_{s} D^{l}(R)_{st} \mu_{s} k_{ls}(x) .
  \label{9-33}
  \end{equation}
  % % %
It can be concluded that the representation $D^{l}(R)_{st}$ that was obtained in equation \eqref{9-19} is also the representation generated by the harmonic polynomials $\mu_{s} k_{ls}(x)$. We will, henceforth, consider these functions rather than the functions $f$ and $g$ defined in equations \eqref{9-28} and \eqref{9-29},
from which they differ by an undetermined factor.

The coordinates of the point $x$ can be expressed in spherical polar coordinates as $x = r \sin \theta \cos \phi$, $y = r \sin \theta \sin \phi$, $z = r \cos \theta$. If the functions $\mu_{m} k_{lm}(x)$ are written in terms of these coordinates it is evident that they have the form $r^{l} Y_{lm}\paren{\theta,\phi}$ where $Y_{lm}\paren{\theta,\phi}$ is a polynomial in $\sin \theta$, $\cos \theta$, $\sin \phi$, {\rd{and}} $\cos \phi$. The functions $Y_{lm}\paren{\theta,\phi}$ are the important {\bf{spherical harmonics}}. Since $r^{l}Y_{lm}\paren{\theta,\phi}$ must satisfy Laplace's equation in spherical polar coordinates, the spherical harmonics must satisfy
  % % % EQUATION
  \begin{equation}
    \frac{1}{\sin \theta} \pd{}{\theta} \sin \theta \pd{Y_{lm}}{\theta} + \frac{1}{\sin^{2} \theta}\pdt{Y_{lm}}{\phi} + l(l+1)Y_{lm} = 0 .
    \label{9-34}
  \end{equation}
  % % %
It will now be shown that for fixed $l$ the spherical harmonics are
determined up to an arbitrary constant by equation \eqref{9-33}. We denote
by $\paren{\theta, \phi}$ and $\paren{\theta', \phi'}$ the angular coordinates of the points $x$ and $R^{-1}x$ respectively. In terms of the spherical harmonics \eqref{9-33} becomes
  % % % EQUATION
  \begin{equation}
    Y_{lm}\paren{\theta', \phi'} = \sum_{n} D^{l}(R)_{nm} Y_{ln} \paren{\theta, \phi}.
    \label{9-35}
  \end{equation}
  % % %
If $\R$ is a rotation about the z axis by a the angles $\paren{\theta', \phi'}$ are simply $\paren{\theta, \phi-\alpha}$. USing equation \eqref{9-23a}, we can write \eqref{9-35} as
  % % % EQUATION
  \begin{equation*}
    Y_{lm}\paren{\theta, \phi-\alpha} = e^{i m \alpha} Y_{lm} \paren{\theta, \phi}.
  \end{equation*}
  % % %
Putting $\phi = 0$ and changing the sign of $\alpha$, we obtain
  % % % EQUATION
  \begin{equation}
    Y_{lm}\paren{\theta, \alpha} = e^{i m \alpha} Y_{lm} \paren{\theta, 0} ,
    \label{9-36}
  \end{equation}
  % % %
indicating that the only dependence of $Y_{lm}$ on the azimuthal angle is in the factor $e^{i m \alpha}$.

In the direction of the positive $z$ axis the spherical harmonics must be independent of the azimuthal angle, that is, $Y_{lm}\paren{0, \alpha} = Y_{lm}\paren{0, 0}$. If $m\ne0$, however, this can only be the case if $Y_{lm}\paren{0, 0} = 0$ in view of the known dependence of $Y_{lm}$ on the azimuthal angle. On the other hand $Y_{l0}\paren{0, 0} \ne 0$ since otherwise the spherical harmonics would vanish identically. The value of $Y_{l0}\paren{0, 0}$ will be chosen arbitrarily to be 1 so that
  % % % EQUATION
  \begin{equation}
    Y_{lm}\paren{0, \phi} = \delta_{m0} .
    \label{9-37}
  \end{equation}
  % % %
We now put $\theta = \phi == 0$ in \eqref{9-35} and replace $\R$ by $\R^{-1}$; the result is, since $\D^{l}\paren{\R^{-1}}_{nm} = \D^{l}\paren{\R}_{nm}^{*}${\rd{,}}
  % % % EQUATION
  \begin{equation*}
    Y_{lm}\paren{\theta', \phi'} = \D^{l}\paren{\R}_{m0}^{*} .
  \end{equation*}
  % % %
where $\theta'$, $\phi'$ are the polar angles of the direction into which $\R$ rotates the $z$ axis. The rotation $\Z\paren{\alpha + \pi/2} \X\paren{\beta}$ is known from the definition of the Euler angles to rotate the vector $k$ parallel to the $z$ axis into the direction whose angular coordinates are $\paren{\beta, \alpha}$. We can, therefore, write,	
  % % % EQUATION
  \begin{equation}
    Y_{lm}\paren{\beta, \alpha} = \D^{l}\paren{\alpha+\frac{\pi}{2},\beta,0}_{m0}^{*}
    \label{9-38}
  \end{equation}
  % % %
or
  % % % EQUATION
  \begin{equation}
    Y_{lm}\paren{\theta, \phi} = e^{i m \phi} d^{l}_{m0}(\theta) .
    \label{9-39}
  \end{equation}
  % % %
This result will be applied to obtain various properties of the spherical harmonics.


\endinput  % ---------------------------------------------------------------------