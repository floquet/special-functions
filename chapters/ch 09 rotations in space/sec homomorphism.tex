\section{The homomorphism of $SU(2)$ onto $O(3)^{+}$}

The Lie algebra of $O(3)$ is known to be composed of all skew-symmetric $3 \times 3$ matrices with real elements. A matrix of this form can be expressed as a linear combination of the matrices $\mathbf{i}$, $\mathbf{j}$, $\mathbf{k}$ of equation (4.50). Since these matrices satisfy the same com- mutation relations (up to a factor) as the matrices $\I_{x}$, $\I_{y}$, $\I_{z}$ of (8.7) the groups $O(3)$ and $SU(2)$ must be locally isomorphic. It will now be shown that there is a homomorphic mapping of $SU(2)$ onto the group $O(3)^{+}$.

We consider for each point in space with coordinates $(x, y, z)$ the matrix
\begin{equation}
  \PP(x, y, z) = \mat{cc}{
  i{\color{red}z} & ix - y \\
  ix + y & -i z }
  \label{eq:09:P}
\end{equation}

The matrix $\PP(x, y, z)$ can be expressed in the form $xI_{x} + yI_{y} + zI_{z}$ where $\I_{x}$, $\I_{y}$, and $\I_{z}$ are the matrices defined in equation (8.7). It is observed that $\PP$ is a skew-Hermitian matrix and has determinant $x^{2} + y^{2} + z^{2}$. It is also true that any $2 \times 2$ skew-Hermitian matrix with zero trace can be expressed in the form \eqref{eq:09:P} for a suitable choice of $x, y, z.$

We consider now, for an arbitrary matrix $\A \in SU(2)$ the matrix $\PP'$ defined by
\begin{equation}
  \PP' = \A\, \PP'\, \A^{-1} = \A' \, \PP' \, \A^{\dagger}
  \label{eq:916}
\end{equation}
It is observed that $\PP'^{\dagger} = \A \PP^{\dagger} \A^{\dagger} = -\A \PP \A^{\dagger} = - \PP^{\dagger}$ so that $\PP'$ is skew-Hermitian.	Furthermore, Tr$\paren{\PP'}$ = Tr$\paren{\PP'}$ = 0 by the trace invariance property. It is, therefore, possible to write
\begin{equation}
  \PP' = \mat{cc}{
  i z' & ix' - y' \\
  ix' + y' & -i z' } .
  \label{eq:09:P'}
\end{equation}
It follows from the form of (16) that the numbers $x'$, $y'$, $z'$ are linear functions of $x$, $y$, $z$. It can, furthermore, be seen that the determinant of $\PP'$ is equal to the determinant of $\PP$, since $\abs{\A\PP\A^{-1}} = \abs{\A} \abs{\PP} \abs{\A^{-1}} = \abs{\PP}$. This result implies that
\begin{equation}
  x'^{2} + y'^{2} + z'^{2} = x^{2} + y^{2} + z^{2}
\end{equation}
and that the linear transformation on $x$, $y$, $z$ generated by \eqref{eq:916} is in fact a rotation. We conclude that for each $\A \in SU(2)$ there is a corresponding rotation $f\paren{\A} \in 0(3)^{+}$.

The mapping $f$ is a homomorphism since, if $\A_{1}$ and $\A_{2}$ are any two elements of $SU(2)$, $f\paren{\A_{1}, \A_{2}} $ is the rotation generated by transforming $\PP$ to 
\begin{equation*}
  \A_{1}\A_{2} \PP \paren{\A_{1}\A_{2}}^{-1} = \A_{1} \paren{\A_{2} \PP \A_{2}}^{-1} \A_{1}^{-1}.
\end{equation*}
This is, however, the transformation generated by transforming $\PP$ first by $\A_{2}$ and then by $\A_{1}$; the resulting rotation is that obtained by rotating first by $f(\A_{2})$ and then by $f(\A_{1})$ or $f(\A_{1}\A_{2}) = f(\A_{1}) f(\A_{2})$.???

It will now be shown that each proper rotation is the image under $f$ of the same element in $SU(2)$. This is proved by exhibiting explicitly elements of $SU(2)$ that generate rotations by an arbitrary angle about the $x$ and $z$ axes. We will require the following multiplication laws of the matrices $\I_{x}$, $\I_{y}$, $\I_{z}$.
\begin{equation}
  \begin{split}
    \I_{x} \, \I_{y} &= -\I_{y} \, \I_{x} = \I_{z} \\
    \I_{y} \, \I_{z} &= -\I_{z} \, \I_{y} = \I_{x} \\
    \I_{z} \, \I_{x} &= -\I_{x} \, \I_{z} = \I_{y} .
  \end{split}
\end{equation}
We consider now the matrix
\begin{equation}
  \A \paren{0, \phi, 0}  = \mat{cc}{
    e^{i \phi} & 0 \\ 0 & e^{-i \phi}
  } = \cos \phi + \sin \phi \I_{z} .
\end{equation}
It is apparent that this matrix commutes with $\I_{z}$ so that
\begin{equation}
  \A \paren{0, \phi, 0} \PP \,  \A \paren{0, \phi, 0}^{\dagger} = x' \I_{x} + y' \I_{y} + z' \I_{z}
\end{equation}
Since $z$ is unchanged, $\A \paren{0, \phi, 0}$ evidently generates a rotation about the z axis. We can calculate explicitly
\begin{equation*}
  \begin{split}
    \paren{\cos \phi + \sin \phi \I_{z}} &\paren{x \I_{x} + y \I_{y} + z \I_{z}} \paren{\cos \phi - \sin \phi \I_{z}} \\
    &= z \I_{z} + \paren{ \paren{\cos^{2}\phi - \sin^{2}\phi}x - \paren{2 \sin \phi \cos \phi} y } \I_{x} \\
    &\phantom{= z \I_{z}\ } + \paren{ \paren{2 \sin \phi \cos \phi} x + \paren{\cos^{2}\phi - \sin^{2}\phi}y} \I_{y} \\
    &= \paren{ x \cos 2 \phi - y \sin 2 \phi} \I_{x} 
     + \paren{ x \sin 2 \phi - y \cos 2 \phi} \I_{y} 
     + z \I_{z} .
  \end{split}
\end{equation*}
This result indicates that
\begin{equation*}
  \begin{split}
    x' &= x \cos 2 \phi - y \sin 2 \phi , \\
    y' &= x \sin 2 \phi - y \cos 2 \phi , \\
    z' &= z .
  \end{split}
\end{equation*}
and that $\A \paren{0, \phi, 0}$ generates a rotation about the $z$ axis by $2\phi$. It can be shown in the same way that the matrix
\begin{equation}
  \A \paren{\theta, 0, 0}  = \mat{cc}{
  \cos \theta & i \sin \theta \\ i \sin \theta & \cos \theta
  } =
  \cos \theta \, \I + \sin \theta \, \I_{x}
\end{equation}
generates a rotation by $2\theta$ about the $x$ axis. Since an arbitrary proper rotation $R \paren{\alpha, \beta, \gamma} $ can be expressed in the form $Z\paren{\alpha}X\paren{\beta}Z\paren{\gamma}$, it can be generated by the element $\A \paren{0, \alpha / 2, 0} \A \paren{0, \beta / 2, 0} \A \paren{0, \gamma / 2, 0} $ in $SU(2)$. This element can be written
\begin{equation}
  \begin{split}
%
    &\phantom{=\ } 
      \mat{cc}{ e^{i \alpha / 2} & 0 \\ 0 & e^{-i \alpha / 2} }
      \mat{cc}{ \cos \frac{\beta}{2} & 
              i \sin \frac{\beta}{2} \\ 
              i \sin \frac{\beta}{2} & 
                \cos \frac{\beta}{2} }
      \mat{cc}{ e^{i \gamma / 2} & 0 \\ 0 & e^{-i \gamma / 2} } \\
%
    &= 
      \mat{cc}{ e^{ i \paren{\alpha + \gamma} / 2} \cos \frac{\beta}{2} &
               ie^{ i \paren{\alpha - \gamma} / 2} \sin \frac{\beta}{2} \\ 
               ie^{-i \paren{\alpha - \gamma} / 2} \sin \frac{\beta}{2} &
                e^{-i \paren{\alpha + \gamma} / 2} \cos \frac{\beta}{2} } \\
%
    &= \A \paren{ \tfrac{\beta} {2}, \tfrac{\alpha + \gamma} {2}, \tfrac{\alpha - \gamma} {2} } 
  \end{split}
\end{equation}


It is important to observe that the mapping $f$ of $SU(2)$ onto $0(3)^{+}$ is not an isomorphism; since $\A \PP \A^{-1} = \paren{-\A}  \PP \paren{-\A} ^{-1}$ it is immediately apparent that $f\paren{\A} = f\paren{-\A}$ and that the mapping $f$ cannot be isomorphic. It is important to calculate the kernel of $f$, that is, the set of elements	$\A \in SU(2)$ that satisfy $f\paren{\A} = \I$. In order that $f\paren{\A} = \I$, it is necessary that $\A$ satisfy $\A \PP \A^{-1} = \PP$ or $\A\PP = \PP\A$ for all matrices $\PP$ of the form \eqref{eq:09:P}. In particular, the matrices $\I_{x}$ and $\I_{z}$ must commute with $\A$. It can be observed from (8.6) and (8.7) that an arbitrary matrix $\A$ can be written 
\begin{equation*}
  \A =
  \cos \theta \, \cos \phi + 
  \cos \theta \, \sin \phi \, \I_{z} + 
  \sin \theta \, \cos \psi \, \I_{x} +
  \sin \theta \, \sin \psi \, \I_{y}
\end{equation*}
In order that this commute with $\I_{z}$ it is necessary that
\begin{equation}
  \sin \theta \, \cos \psi \, \I_{y} + \sin \theta \, \sin \psi \, \I_{x} = 0
\end{equation}
which implies $\sin \theta = 0$ and $\cos \theta = 1$. The requirement that $\cos \phi + \sin \phi \, \I_{z}$ commute with $\I_{x}$ implies that $\sin \phi = 0$ and $\cos \phi = \pm1$. We can conclude that the kernel of the homomorphism consists of the matrices $\I$ and $-\I$. This subgroup is denoted by $Z_{2}$. To conclude this discussion, we can state that the group $0(3)^{+}$ is isomorphic to the factor group $SU(2)/Z_{2}$.	The elements of this group are the cosets of $Z_{2}$ consisting of pairs of elements of $SU(2)$ which differ only in sign.

\endinput