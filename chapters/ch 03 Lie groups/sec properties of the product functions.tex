\section{Properties of the Product Functions}


The following discussion will contain the calculation of derivatives of the product functions at various points in the group. The customary notation for partial derivatives can be rather misleading, so that we will introduce and adhere to the following convention. The symbol Ofa/or!!will be used to represent the derivative of the product function fa with respect to the 11th coordinate of the first set
of arguments of fa, regardless of the point at which this function is actually evaluated. Similarly, ofa /8 q Y denotes the derivative of fa with respect to the yth coordinate of the second argument. The summation convention of summing over repeated indices will also be used.
We will now discuss a few properties of the derivatives of the product functions. The first of these are rather trivial and are obtained by differentiating (1) with respect to p f3:
  \begin{equation}   %  =   =   =   =   =
   %\begin{split}
      a + b
   %\end{split}
   %\label{eq:}
  \end{equation}
These imply further that
  \begin{equation}   %  =   =   =   =   =
   %\begin{split}
      b + c
   %\end{split}
   %\label{eq:}
  \end{equation}
If rO' =P (p, q), it is possible to show that the Jacobian
  \begin{equation*}   %  =   =   =   =   =
   %\begin{split}
      c + d
   %\end{split}
   %\label{eq:}
  \end{equation*}
is nonzero. The parameters of the inverse of the element with the
r ,r ,??.,r) of0'
12 =--(pq)
parameters p will be denoted by i1 (p), i2(p), ??., in (P) so that 
  \begin{equation}   %  =   =   =   =   =
   %\begin{split}
      d + e
   %\end{split}
   %\label{eq:}
  \end{equation}
Consider then the derivative with respect to rJ3 of the identity fO'(f(p,q),1(q�=pO', which is a reflection of the identity abb-1 = a. The derivative is
  \begin{equation}   %  =   =   =   =   =
   %\begin{split}
      e + f
   %\end{split}
   %\label{eq:}
  \end{equation}
This shows that the matrix whose elements are of Y/op(3 has an inverse and hence that the Jacobian is nonzero. It is possible to show in a similar way that
  \begin{equation*}   %  =   =   =   =   =
   %\begin{split}
      f + g
   %\end{split}
   %\label{eq:}
  \end{equation*}
These results are simply reflections of the fact that the transformations r(p) = f(P, q) with q fixed, and r(q) = f(P, q) with p fixed, are necessarily invertible.
It is now possible to calculate the derivatives of the functions 1(P) that are determined bl the inversion operation. Consider the derivative with respect to p of the identity (5):
  \begin{equation*}   %  =   =   =   =   =
   %\begin{split}
      e = ma
   %\end{split}
   %\label{eq:}
  \end{equation*}
This equation can be solved for oi Y/op(3 by using equation (6) which instructs us how to invert the matrix whose elements are of0'/ op(3 ? The desired result is
  \begin{equation}   %  =   =   =   =   =
   %\begin{split}
      2 + 2
   %\end{split}
   %\label{eq:}
  \end{equation}
It is apparent that higher derivatives of i could be calculated by differentiation of this result.
In the next section we will have occasion to use functions vO!f3 defined by
  \begin{equation}   %  =   =   =   =   =
   %\begin{split}
      2 - 2
   %\end{split}
   %\label{eq:}
  \end{equation}
These functions satisfy an identity which can be derived by differentiating equation (2), the associative law, with respect to r" and then setting r =O. Using the chain rule and (1) gives
  \begin{equation*}   %  =   =   =   =   =
   %\begin{split}
      2 + 1
   %\end{split}
   %\label{eq:}
  \end{equation*}
or, from (8),
  \begin{equation}   %  =   =   =   =   =
   %\begin{split}
      a - c
   %\end{split}
   %\label{eq:}
  \end{equation}
This equation will play an important role in Chapter 5 when an invariant integral over the group will be defined.

\endinput  % ---------------------------------------------------------------------