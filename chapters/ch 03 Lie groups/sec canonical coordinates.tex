\section{Canonical Coordinates}

It has been seen that a solution to \eqref{eq:3-13} exists for every tangent vector a so that a one-parameter subgroup passes through e in every direction. It will now be shown that there is a neighborhood V of e such that a one-parameter subgroup passes through each point in V. It is first noted that if p(t) is a one-parameter subgroup with tangent vector a, p(ct) is a one-parameter subgroup with tangent vector ca. We will now show the dependence of p(t) on a explicitly by writing p(t) = z(a, t) where a = p(O). Then the above observation shows that
z(a, ct) = z(ca, t). Consider now the function w(a) = z(a, 1). It can be     from the above identity that zO!(a, t) = wO!(ta). If this identity is differentiated -with respect to t we find
  \begin{equation*}   %  =   =   =   =   =
   %\begin{split}
      partial
   %\end{split}
   %\label{eq:}
  \end{equation*}
from the differential equation (13). The theory of ordinary differential equations guarantees that the derivatives 8 wCJaaf3 exist. Putting t =0 now shows that
  \begin{equation*}   %  =   =   =   =   =
   %\begin{split}
      partial
   %\end{split}
   %\label{eq:}
  \end{equation*}
and, since the numbers af3 can be chosen arbitrarily, that 
  \begin{equation*}   %  =   =   =   =   =
   %\begin{split}
      partial
   %\end{split}
   %\label{eq:}
  \end{equation*}

It can be concluded that the Jacobian of the transformation w(a) is 1 at a = O. The implicit function theorem can now be used to prove that there are open regions V in the parameter space and U in the space of tangent vectors such that the equation p =w(a) can be solved to give a as a single-valued function of p. In other words, for every p � V, there is a unique a � U such that p =w(a). It is therefore possible to use the components of the tangent vectors in U as a new set of group parameters characterizing the group element with old coordinates p =w(a), provided p � V. The new coordinates of a group element are the components of the tangent at e to the one-parameter subgroup that passes through the point at t =1. This new coordinate system is said to be \uline{canonical}.

The one-parameter subgroups have the very simple parametric equations a(t) =at as long as at is in U. This is proved as follows. Let p(t) be the old coordinates of a one-parameter subgroup with tangent vector a � U; then p(l) has new coordinates a. Consider the curve ps in the old parameters defined by ps(t) =p(st). The tangent at 0 to ps is sa and ps (1) has new coordinates sa (provided sa � U). On the other hand ps (1) = p(s), showing that the new coordinates of p(s) are simply sa, provided p(s) � V (or sa � U).

It is also possible to prove that any continuous one-parameter subgroup (the arc going through the unit element), not assumed differentiable, must have a parametric equation of the form p(t) =at for some a[2]. Therefore, the coordinates must actually be differentiable functions of t in the open region V and must satisfy equation (13) ? The proof of this fact will be omitted.

This has the following consequence that will be useful in the next chapter. Suppose that f is a homomorphic mapping of a Lie group G onto another Lie group G'. If a(t) is a one-parameter subgroup in G, f(a(t� is a one-parameter subgroup of G'. Iffurther, the coordinates of G and G' are canonical, the coordinates of a(t) and f(a(t� are expressible as at and a't respectively. It will be assumed that a and a' are in the neighborhoods V and V' in which canonical coordinates for G and.G' are defined; this can always be achieved by a change of scale in the parameter t. If f is regarded as a mapping from the coordinates of G to the coordinates of G' , we can write
  \begin{equation}   %  =   =   =   =   =
   %\begin{split}
     fat  3-17
   %\end{split}
   \label{eq:3-17}
  \end{equation}
where a,a is independent of t but depends on a.
It can be proved that the fa are linear functions of their arguments.
It is first necessary to prove that they are differentiable at the origin. To show this we put Jl = oJJ.v so that (17) becomes
  \begin{equation*}   %  =   =   =   =   =
   %\begin{split}
      1
   %\end{split}
   %\label{eq:}
  \end{equation*}
Differentiation with respect to t shows that afajOav (0) exists and is a,a(O, 0, ...,1, ...,0), the 1 being the vth argument of a,a.
The derivative of \eqref{eq:3-17} with respect to t, evaluated at t = 0, is 
  \begin{equation*}   %  =   =   =   =   =
   %\begin{split}
      d f a
   %\end{split}
   %\label{eq:}
  \end{equation*}
On the other hand, if we put t =1 in (17), we find that
  \begin{equation*}   %  =   =   =   =   =
   %\begin{split}
      f a a
   %\end{split}
   \label{eq:3-18}
  \end{equation*}
Equation (18) shows that f is a linear functions of a.

The transformation to canonical coordinates from an arbitrary
parametrization is differentiable and invertible in the neighborhood of the origin. From this it can be shown, in view of the differentiability of \eqref{eq:3-18}, that a homomorphic mapping of one Lie group onto another must be a differentiable mapping of the coordinates in the neighborhood of the origin. In particular, two parametrizations in
a Lie group must be connected by a differentiable transformation.
In this case the transformation must also be invertible, since in each parametrization there are neighborhoods of the origin in which the points can be placed in one-to-one correspondence with the group elements, and therefore with each other.


\endinput  % ---------------------------------------------------------------------