\chapter{Lie Groups}

%%

Although the study of group theory originated with finite groups, the theory of Lie groups has achieved greater importance than that of finite groups, and it is Lie groups that are of primary significance for our purposes. Whereas the elements of a finite group form a discrete set, those of a Lie group form a continuum; they depend differentiably on one or more parameters. Because of this, topological considerations, the concepts of neighborhood, open sets, closed sets, and so on, play an important role in the basic theory of Lie groups. Many remarkable results, for example that continuity of the group implies differentiability of the group under certain circumstances, have been established for Lie groups. However, in the present discussion we will avoid involved topological considerations as much as possible, often by making more restrictive assumptions than are necessary (differentiability instead of continuity, and so on).

Many of the properties of a Lie group are determined by the elements in an arbitrarily small neighborhood of the identity, since such a neighborhood can be multiplied by itself repeatedly to generate at least a large part of the group. For example, if the elements in any neighborhood of the identity commute, the group is Abelian. There is no corresponding situation in finite groups, in which a small subset cannot be expected to determine most of the group properties.

Whereas the theory of finite groups can stand alone, without relying on any other part of mathematics, the theory of Lie groups often makes extensive use of the theory of ordinary and partial differential equations. In addition, many Lie groups have connectedness properties which, if perhaps not complicated, are still different from the connectedness properties of ordinary Euclidean spaces. An example of this, that of the group of rotations in two dimensions, will be found in the next section. There are Lie groups with even more complicated ``topology in the large." As a result of these circumstances, particularly the reliance on the theory of differential equations, the theory of Lie groups is much less independent of other parts of mathematics than is the theory of finite groups.
%
\input{chapters/"ch 03 Lie groups"/"sec Lie groups"}
\input{chapters/"ch 03 Lie groups"/"sec compact groups"}
\input{chapters/"ch 03 Lie groups"/"sec locally isomorphic groups"}
\input{chapters/"ch 03 Lie groups"/"sec properties of the product functions"}
\input{chapters/"ch 03 Lie groups"/"sec one parameter subgroups"}
\input{chapters/"ch 03 Lie groups"/"sec canonical coordinates"}
\input{chapters/"ch 03 Lie groups"/"sec one parameter subgroups of matrix groups"}

\endinput