\section{One Parameter Subgroups of Matrix Groups}
To provide an illustration of the concept of one-parameter subgroup, these subgroups of the group L(n) of n x n nonsingular matrices will be calculated. A parametrization of     grq"!lP is obtained by expressing each matrix element in the form 0 1) + pI) so that the coordinates of the identity are zero. Rather than attempt to use a single index, it is much more convenient to use a pair of indices, namely, (ij), i = 1, 2, ... , n, j = 1, 2, ... , n to label the parameters.
The product functions for the matrix group are then given by 
  \begin{equation}   %  =   =   =   =   =
   %\begin{split}
      3-19
   %\end{split}
   %\label{eq:}
  \end{equation}
From this we obtain
  \begin{equation}   %  =   =   =   =   =
   %\begin{split}
      3-20
   %\end{split}
   %\label{eq:}
  \end{equation}

The differential equation \eqref{3-13} is therefore
  \begin{equation*}   %  =   =   =   =   =
   %\begin{split}
      3-21
   %\end{split}
   %\label{eq:}
  \end{equation*}
where the amn are the matrix elements of the "tangent" matrix A. c
This equation can be conveniently written in matrix form as
  \begin{equation}   %  =   =   =   =   =
   %\begin{split}
      \dot{P} = A + PA
   %\end{split}
   %\label{eq:}
  \end{equation}

It can be verified that this equation has the solution 
  \begin{equation*}   %  =   =   =   =   =
   %\begin{split}
      3-23
   %\end{split}
   %\label{eq:}
  \end{equation*}
where
  \begin{equation*}   %  =   =   =   =   =
   %\begin{split}
      3-24
   %\end{split}
   %\label{eq:}
  \end{equation*}
is a convergent (for all t) power series solution of \eqref{3-22}. (The largest matrix element of Ar.is smaller in absolute value than (nAm)r where Am is the absolute value of the largest matrix element of A.) We can conclude that the one-parameter subgroups of L(n) can be written in the form eAt where A is an arbitrary n x n matrix.


\endinput  % ---------------------------------------------------------------------