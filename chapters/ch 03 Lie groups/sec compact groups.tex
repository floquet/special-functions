\section[Compact Groups]{Compact Groups}

The theory of representations has particularly simple results for a certain class of groups, the compact groups. The property of compactness also simplifies the derivation of some general theorems.
In this section we will define a compact group.
A simple example of a compact (not group theoretical) set is a closed finite interval on the real line. Such an interval, [a, bJ, has two important properties. The first is that for any infinite set of points { xi} , a :s Xi :s b, there is (at least) one limit point also in
[a, b]. If every neighborhood of a point x contains an infinite number of the points Xi, x is said to be a \uline{limit element} of the set {xi}' The second property of a closed finite interval could be called the Heine-Borel property. This stipulates that from any set of open intervals Ia = (xa , Ya) that covers [a, b], it is possible to select a finite subset of intervals Ia1 ' Ia2 ' ??? , Ian that again cover [a, b]. The verb "cover" means in this case that each element of [a, b] is also in one of the covering open intervals. The classic Heine-Borel theorem is that the second property is a consequence of the first property.
A Lie group will be defined to be compact if there is a parametrization in which the group is covered by coordinates in a finite number of bounded parameter domains. It is essential, of course, that the parametrization be such that the differentiability and closure properties described above are satisfied. A bounded parameter domain is one that can be enclosed in a rectangle in the coordinate space. Specifically, there are numbers ai, bi, i = 1, 2, ..., n such that
It ai < pi < bi for the coordinates pi of points in the domain.
Let  be an infinite set of elements of a compact Lie group.
)int
1
3
ain ,s
any
t,
Y 10
Then one of the parameter domains must also contain the coordinates of an infinite number of the gn and these coordinates must have a limit element either in or on the boundary of the parameter domain. (The proof can be found in many books on advanced calculus.) The closure property of the group then assures that the limit element is
a group element. The first property of closed finite intervals, that any infinite sequence of group elements has a limit element, is therefore satisfied by a compact Lie group.
It is also possible to adapt the usual proof of the Heine-Borel theorem to show that a compact Lie group G has the second property of closed finite intervals in the following sense. Suppose a family of open regions Ua covers G by including the coordinates of each element of G from at least one parameter domain. It is then possible to select a finite number of the Ua that cover G in the same way. (The definition is complicated here by the possibility that a given group element may have coordinates in several parameter domains.)

\endinput  % ---------------------------------------------------------------------