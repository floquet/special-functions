\chapter{Abstract groups}

It seems natural to commence the study of the application of group theory to the special functions with a review of group theory itself. This chapter is included to meet this possible need and to define some of the terms that are met in the remainder of the book. The contents of this chapter will be familiar, at least in outline, to many readers, and they are .invited either to omit it or to take pleasure in the review of familiar concepts. This chapter is for this reason brief and rather tedious, without the examples and observations which give rise to the interest in the subject. 

It is instructive, in the study of group theory, to view it as an investigation of the 1$-$1 mappings of a set $S$ onto itself. The group elements are functions $f$ defined for all $x \in S$ such that $f(x) \in S$, that is, the domain and range of $f$ are both $S$. The function $f$ must be 1$-$1; that is, for each $y \in S$, there must be a unique $x$ such that $y = f(x)$. A group composed of such mappings is called a \emph{transformation group}. As an example, the set $S$ might consist of the first $n$ integers, in which case the group is the permutation group on $n$ objects.

The requirement that the mapping be 1$-$1 implies that it has a well-defined inverse $f^{-1}$; $f^{-1}$ is the mapping that undoes the operation of $f$.

If $f$ and $g$ are 1$-$1 mappings, the composition mapping $gf$ defined by $[gf] (x) = g(f(x))$ is easily seen to be 1$-$1. This mapping is called the product of $g$ and $f$. It is this type of multiplication that is the basic operation in group theory, adjoining an element, the product $gf$, to each ordered pair of elements $g$ and $f$.

Some groups are sets of mappings of a set onto itself that are restricted in some further way than the 1$-$1 condition that has been imposed. For example, the group of transformations of a plane into itself that keep the distance between points fixed and also keep one point fixed is the rotation group in the plane. It is with groups of this nature that we will be principally concerned.

%%% sections
\input{chapters/"ch 02 abstract groups"/"sec abstract groups"}
\input{chapters/"ch 02 abstract groups"/"sec subgroup and factor groups"}
\input{chapters/"ch 02 abstract groups"/"sec homomorphisms"}
\input{chapters/"ch 02 abstract groups"/"sec further aspects"}


\endinput