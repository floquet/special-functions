\section{Some further aspects of groups}


Our review of some of the concepts from abstract group theory will be concluded in this section with discussions of miscellaneous topics: equivalence classes in groups, direct products of groups, and two important subgroups. The discussion will be limited to definitions and results more or less pertinent to the sequel.

An important classification of elements within a group $G$ is provided by the conjugacy or equivalence classes in the group defined as follows. An element $b \in G$ is said to be \emph{equivalent}, or \emph{conjugate}, to $a\in G$ if there is some element $t$ such that $tat^{-1}=b$. If this is the case, $t$ is said to \emph{transform} $a$ into $b$. Since $e$ transforms $a$ into $a$, $a$ is equivalent to itself, and since $t^{-1}$ transforms $b$ into $a$, $a$ is equivalent to $b$. Furthermore, if $c$ is equivalent to $b$, that is, $c = sbs^{-1}$, and $b$ is equivalent to $a$, $b = tat^{-1}$ , then $c$ is equivalent to $a$ since $c = (st)a(st)^{-1}$ .	The group $G$ can be divided into disjoint subsets with the property that any two elements common to one subset are equivalent and any two elements from different subsets are not equivalent; these subsets are called the \emph{equivalence} or \emph{conjugacy classes} of $G$. The class to which any element $a$ belongs will be denoted by $C_{a}$. As an example, we note that $C_{e}$ consists only of $e$ since $tet^{-1} = e$ for all $t \in G$. If $G$ is Abelian, it is clear that each class consists of only a single element. 

It is worth while to observe that a subgroup $N$ of $G$ is normal if and only if it is composed of entire classes. This follows since, if $a \in  N$,$tat^{-1} \in N$ if $N$ is normal, so that $C_{a} \subset N$. On the other hand, if $C_{a} \subset N$ for each $a \in N$, $tNt^{-1} \subset N$ and $N$ is normal. The subgroup $N$ can itself be regarded as a group composed of classes $\overline{C}$ with respect to $N$. Two elements $n_{1}$ and $n2$ are in the same class $C$ if there is an element $t \in  N$ such that$ tn_{1}t^{-t} = n_{2}$; it follows that if two elements are in the same class $\overline{C}$ of $N$, they are in the same class $C$ of $G$. On the other hand, two elements common to a class $C$ in $G$ may be in different classes in $N$, since the element which transforms one to the other may be outside of $N$. We can conclude from this discussion that each class $C$ in $G$ that is contained in $N$ is composed of entire classes $\overline{C}$ of $N$.

If two groups $G_{1}$ and $G_{2}$ are known, it is possible to form a third group from them. The \emph{direct product} $G_{1} \otimes G_{2}$ of the two groups is defined to be the set of all pairs of the form $(g_{1}, g_{2})$ where $g_{1} \in  G_{1}$ and $g_{2} \in  G_{2}$� The product of two such pairs is defined in a rather obvious way by
\begin{equation}
  \paren{g_{1} g_{2}} \paren{h_{1}, h_{2}} = \paren{g_{1}h_{1}, g_{2}h_{2}} .
\label{eq:2.5}
\end{equation}
The identity in $G_{1} \otimes	G_{2}$ is $\paren{e_{1}, e_{2}}$ and the inverse of $\paren{g_{1}, g_{2}}$ is $\paren{g_{1}^{-1}, g_{2}^{-1}}$.

The direct product has no interesting algebraic structure other than that of each of its factors. A question of some interest is: given a group $G$, does it have subgroups $G_{1}$ and $G_{2}$ such that $G$ is (isomorphic to) the direct product of $G_{1}$ and $G_{2}$? It will now be shown that if $G$ has normal subgroups $N_{1}$ and $N_{2}$ such that $N_{1} \cap N_{2} = \{e\}$ and $N_{1}N_{2} = G$, then $G$ is isomorphic to $N_{1} \otimes N_{2}$. It will first be shown that if $n_{1} \in  N_{1}$ and $n_{2} \in  N_{2}$, then $n_{1}n_{2} = n_{2}n_{1}$. This is proved by considering the \textbf{commutator} $q = n_{1}n_{2}n_{1}^{-1}n_{2}^{-1}$ of $n_{1}$ and $n_{2}$. Since $N_{2}$ is a normal subgroup, $n_{1}n_{2}n_{1}^{-1} = n_{2}' \in  N_{2}$ and $q = n_{2}'n_{2}^{-1} \in  N_{2}$. Similarly, since $N_{1}$ is a normal subgroup, $n_{2}n_{1}^{-1}n_{2}^{-1} = n_{1}' \in Nt$ and $q = n_{1}n_{1}' \in 	N_{1}$. Therefore, $q \in  N_{1} \cap N_{2}$ and $q$ is necessarily $e$. It follows immediately that $n_{1}n_{2} = n_{2}n_{1}$. Since $N_{1}N_{2} = G$, each element $g \in  G$ can be expressed in the form $n_{1}n_{2}$. This representation is, moreover, unique, since if $g = n_{1}n_{2} = m_{1}m_{2}$ where $n_{1}$ and $m_{1}$ are in $N_{1}$ and $n_{2}$ and $m_{2}$ are in $N_{2}$, then $m_{1}^{-1}n_{1} = m_{2}n_{2}^{-1}$. The left-hand side is in $N^{-1}$ the right-hand side is in $N_{2}$, so that each is necessarily $e$ and $m_{1} = n_{1}$, $m_{2} = n_{2}'$. The isomorphism between $N_{1} \otimes N_{2}$ and $G$ is now given by the relation
\begin{equation}
  \paren{n_{1}, n_{2}} \longleftrightarrow n_{1}n_{2} 
\label{eq:2.6}
\end{equation}
This correspondence is an isomorphism, since if $\paren{n_{1}, n_{2}} \longleftrightarrow n_{1}n_{2}$ and $\paren{m_{1}, m_{2}} \longleftrightarrow m_{1}m_{2}$, then 
\begin{equation*}
  \paren{n_{1}, n_{2}} \paren{m_{1}, m_{2}} = \paren{n_{1}m_{1}, n_{2}m_{2}} \longleftrightarrow 
  \paren{n_{1}m_{1}n_{2}m_{2}} = \paren{n_{1}n_{2}} \paren{m_{1}m_{2}}      
\end{equation*}
from the fact that $n_{2}m_{1} = m_{1}n_{2}$. 

The set $Z$ of all elements that commute with every element of a group $G$ is called the \textbf{center} of $G$. It is easy to show that $Z$ is a subgroup of $G$, since if $z_{1}$ and $z_{2}$ are in $Z$ and $g$ is any element of $G$,
\begin{equation*}
  z_{1} z_{2} g = z_{1} g z_{2} = g z_{1} z_{2}
\end{equation*}
and $z_{1}z_{2} \in Z$. It is also clear that $e \in Z$ and that if $z\in Z$, $z^{-1}\in Z$, so that $Z$ is a subgroup. Since $gZ = Zg$ for all $g \in G$, $Z$ is also a normal subgroup of $G$. Any subgroup of $Z$ is also necessarily a normal subgroup of $G$; such a subgroup is called a \textbf{central normal subgroup}.

We have defined the commutator of two elements $a$ and $b$ to be $aba^{-1}b^{-1}$. The \textbf{commutator subgroup} of $G$ is defined to be the set of all elements that can be expressed as a product of commutators. The commutator subgroup is commonly denoted by $G'$. It is clear that the product of two elements of $G'$ is again in $G'$ and that $e \in G'$.
If $q$ is the commutator of $a$ and $b$, $q^{-1}$ is the commutator of $b$ and $a$. It follows from this that the inverse of any element in $G'$ is again in $G'$, and, hence, that $G'$ is a subgroup. To show that $G'$ is a normal subgroup, we observe first that if $q$ is the commutator of $a$ and $b$, $tqt^{-1}$ is the commutator of $tat^{-1}$ and $tbt^{-1}$ so that $t$ transforms any commutator into another commutator. If $x = q_{1}q_{2} \dots q_{n}$ is an element of $G'$, then 
\begin{equation}
  txt^{-1} = \paren{tq_{1}t^{-1}} \paren{tq_{2}t^{-1}} \dots \paren{tq_{n}t^{-1}}   
\end{equation}
is also an element of $G'$, which is, therefore, a normal subgroup.

An interesting property of $G'$ is that $G/G'$ is an Abelian group. Let $aG'$ and $bG'$ be two cosets of $G'$. Their inverses in $G/G'$ can be written $a^{-1}G$, and $b^{-1}G'$ respectively. The commutator of $aG'$ and $bG'$ is 
\begin{equation*}
  \paren{aG'} \paren{bG'} \paren{a^{-1}G'} \paren{b^{-1}G'} = aba^{-1}b^{-1}G' = G',
\end{equation*}
since $\paren{a^{-1}G'} \paren{b^{-1}G'} \in G'$. Since $G'$ is the identity in $G/G'$, $G/G'$ is Abelian.

\endinput