\chapter*{General introduction}


All of us have admired, at one time or another, the theory of the higher transcendental functions, also called special functions of mathematical physics. The variety of the properties of these functions, which can be expressed in terms of differential equations which they satisfy, in terms of addition theorems or definite integrals over the products of these functions, is truly surprising. It is surpassed only by the variety of the properties of the elementary transcendentals, that is the exponential function, and functions derived therefrom, such as the trigonometric functions. At the same time, special functions, as their full name already indicates, appear again and again as solutions of problems in theoretical physics.

These higher transcendentals are analytic functions of their arguments and their properties are usually derived on the basis of their analytic character, using the methods of the theory of analytic functions. Neither the present volume, nor the earlier lectures which gave the incentive to this volume, intend to question the beauty of the analytic theory of the special functions, nor the generality of the results which this theory furnishes. In fact, the lectures started with the observation that the results of the analytic theory are more general than those furnished by the method to be employed in the lectures, thus pointing to a drawback of the considerations to be presented. Though this has been substantially eliminated by subsequent developments, presented also in the present volume and at least partially in other publications, this in no way diminishes the beauty and elegance of the analytic theory, or the inventiveness that was necessary to its development.

Rather, the claim of the present volume is to point to a role of the ``special functions'' which is common to all, and which leads to a point of view which permits the classification of their properties in a uniform fashion. The role which is common to all the special functions is to be matrix elements of representations of the simplest Lie groups, such as the group of rotations in three-space, or the Euclidean group of the plane. The arguments of the functions are suitably chosen group parameters. The addition theorems of the functions then just express the multiplication laws of the group elements. The differential equations which they obey can be obtained either as limiting cases of the addition theorems or as expressions of the fact that multiplication of a group element with an element in the close neighborhood of the unit element furnishes a group element whose parameters are in close proximity of the parameters of the element multiplied. The integral relationships derive from Frobenius' orthogonality relations for matrix elements of irreducible representations as generalized for Lie groups by means of Hurwitz's invariant integral. The completeness relations have a similar origin. Further relations derive from the possibility of giving different equivalent forms to the same representation by postulating that the representatives of one or another subgroup be in the reduced form. Finally, some of the Lie groups can be considered as limiting cases of others; this furnishes further relations between them. Thus, the Euclidean group of the plane can be obtained as a limit of the group of rotations in three-space. Hence, the elements of the representations of the former group (Bessel functions) are limits of the representations of the latter group (Jacobi functions).

Because of the important role which representations of simple groups play in problems of physics, the significance of the ``special functions'' in physical theory also becomes more understandable. In fact, it appears that the elementary transcendentals are also ``special functions,'' the corresponding group being the simplest Lie group of all: the one parametric and hence Abelian Lie group. On the other hand, the field opens up in the opposite direction and one will wonder, when reading this book, what the role and what the properties are of representation coefficients of somewhat more complex Lie groups.
Naturally, the common point of view from which the special functions are here considered, and also the natural classification of their properties, destroys some of the mystique which has surrounded, and still surrounds, these functions. Whether this is a loss or a gain remains for the reader to decide.\\[10pt]

\qquad \qquad \qquad \qquad \qquad \qquad \qquad \qquad \qquad \qquad \qquad \qquad \qquad Eugene P. Wigner


\endinput