\chapter*{Preface}

In theoretical physics, routine use is made of many properties, such as recurrence relations and addition theorems, of the special functions of mathematical physics. These properties are for the most part classical, and their derivations are usually based on the methods of classical analysis. The purpose of this book is to show how these functions are also related to the theory of group representations and to derive their important properties from this theory. This approach elucidates the geometric background for the existence of the relations among the special functions. Moreover, the derivations may be more rationally motivated than are the usual complicated manipulations of power series, integral representations, and so on. I hope that the reader may find in this book reasonably simple derivations of many of the relations commonly used in theoretical physics for which the proofs may otherwise be somewhat unfamiliar.

In order that the book be fairly self-contained, approximately the first third delves into a preliminary discussion of such topics as Lie groups, group representations, and so on. The remaining chapters are devoted to various groups, and the special functions are discussed in conjunction with the group with which it is associated. Because of the inclusion of the introductory material, the only prerequisite is a reasonable knowledge of linear algebra.

The original impetus for the writing of this book was provided by a lecture course given by Professor Eugene P. Wigner a number of years ago. I am greatly indebted to Professor Wigner for his suggestion that I pursue the subject of the lectures further and for his continued friendly interest and advice in the work. I wish to thank Dr. Trevor Luke for carefully checking the manuscript. I also wish to thank my wife, whose encouragement contributed greatly to the writing of the book.

\qquad \qquad \qquad \qquad \qquad \qquad \qquad \qquad \qquad \qquad \qquad \qquad \qquad James D. Talman

London, Canada
July 1968

\endinput