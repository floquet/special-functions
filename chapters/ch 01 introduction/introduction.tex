\chapter{Introduction}

In addition to the elementary transcendental functions such as $e^{x}$ and $\sin x$, an important role is played in mathematical physics by the \emph{special functions}.	Examples of these functions are the Bessel functions, Legendre functions, and hypergeometric functions. For the most part the properties of these functions are studied on the basis of their analytic properties as solutions of ordinary differential equations. For example, the Bessel functions In are the solutions of the differential equation
\begin{equation}
  J_{n}''\paren{x} + \recip{x} J_{n}'\paren{x}  + \paren{1 - \frac{n^{2}}{x^{2}}} J_{n}\paren{x} = 0
  \label{eq:101}
\end{equation}
that behave as $\paren{x_{n}}/\paren{2^{n}n!}$ for $x \to 0$. The Bessel functions are analytic functions of their argument and their order $n$, although in the original form of \eqref{eq:101} the variable $x$ was taken to be real, and the order is frequently restricted to be an integer.

The special functions are treated from this point of view in many excellent books. The best known among physicists are those of Courant and Hilbert [1] and Morse and Feshbach [2] but there are many other treatises, such as those of Rainville [3] and Lebedev [4] devoted to the subject.

The purpose of the present monograph is to demonstrate some of the properties of special functions from the point of view of group theory, or more specifically, from the theory of group representations. It will be seen that many of the special functions are matrix elements, or are simply related to matrix elements, of the representations of elementary groups such as rotation groups and Euclidean
groups. Many properties of the special functions can then be derived from a unified point of view from the group representation property. For example, the Legendre functions are matrix elements of representations of the rotation group in three dimensions. The addition theorems for these functions then follow from the group multiplication law. The differential equations for Legendre functions are a consequence of the differential equations that relate the derivatives of group representations to the corresponding representations of the Lie algebra of the group. The orthogonality and completeness relations are the orthogonality and completeness relations of the group representations. Further relations can be obtained by transforming a given representation to an equivalent form and by reducing the direct product of two representations into a sum of irreducible representations.

The group theoretic treatment shows that the special functions are special only in that they are related to specific groups. The usefulness of group representation theory for the solution of a variety of physical problems makes it natural that representation matrix elements are important special functions for many problems in mathematical physics. It may further be true that the properties that can be derived group theoretically are their most important ones, since they originate from the ``geometric'' properties of the functions.

Although it provides a unified basis for the treatment of special functions, the group theoretic approach has a number of limitations. Not all special functions arise as elements of group representation matrices; for example, no group theoretic basis is known for the gamma and elliptic functions.	The special functions that occur in group representations have restricted indices; for example, only the Legendre functions of the first kind of integer order arise in a natural way. Certain other properties, such as the many integral representations, are not obvious consequences of the representation property.
The special functions that will be considered in detail in this work are the complex exponential function, Jacobi functions and Legendre functions (which are related to hypergeometric functions), Bessel and spherical Bessel functions, Gegenbauer polynomials, associated Laguerre polynomials, and Hermite polynomials. These arise in connection with the groups of pure rotations in two, three, and four dimensions, the Euclidean groups (rigid transformations) in two and three dimensions, and a less familiar group that corresponds to the Lie algebra generated by the position and momentum operators of quantum mechanics.

The approach that we will follow has a certain resemblance to one that has received considerable attention recently and that is related to the \emph{factorization method} of Infeld and Hull [5]. In the factorization method, a single second-order differential equation is replaced, if possible, by a pair of first-order differential equations for a whole set of special functions, that is, a pair of equations of the form
\begin{equation}
  L_{n}^{+} f_{n} = f_{n+1}, \qquad L_{n}^{-} f_{n} = f_{n-1}
  \label{eq:102}
\end{equation}
where $L_{n}^{+}$ and $L_{n}^{-}$ are first-order differential operators. The
second-order equation can then be written in the two alternate forms
\begin{equation}
  L_{n+1}^{-} L_{n}^{+} f_{n} = f_{n}, \qquad L_{n-1}^{+} L_{n}^{-} f_{n} = f_{n} .
  \label{eq:102}
\end{equation}
It is possible to identify the operators $L_{n}^{+}$ and $L_{n}^{-}$ (together with additional operators) with a Lie algebra, and the possible factorizations can be classified by the study of these Lie algebras. The special functions constitute basis functions in representation spaces (as will be defined), for Lie algebras and many of their properties can be obtained in this way. This approach has been thoroughly investigated by W. Miller [6] and B. Kaufman [7].

The approach that will be followed here differs from this in that the primary emphasis will be on groups rather than on the corresponding Lie algebras, and most of the special functions will be related to matrix elements of group representations rather than basis functions in representation spaces, although some interesting results will also be obtained in this way.

A large amount of work has been done in the past few years on the relationship of special functions to group representations, particularly by Miller [8], and a book by N. I. Vilenkin [9] on the subject has been published in Russian. Our attention in this book will be limited for the most part to those properties that seem to be of most interest for mathematical physics. A considerable part of the material
in this book arises from the lecture notes of Wigner on the subject [10].

\endinput