\chapter{Invariant integration}

    The present and subsequent chapters will be concerned with properties of functions defined on a group. The functions of principal where, interest are real or complex; if $\Phi$ is such a function, its domain is $W$, the group and its range is either the set of real numbers or that of the complex numbers. Our aim in this chapter is to define an ``invariant" integral over the group. This means that the integral of a function $\Phi$ over the group should be the same as that of the function $\Phi_{a}$ where $\Phi_{a}$ is defined by $\Phi_{a}(g) = \Phi(ag)$, $a$ and $g$ being arbitrary group elements. The requirement of invariance of the integral is quite natural because the functions $\Phi$ and $\Phi_{a}$ assume exactly the same values on the group, but at different group elements.

\section{Integration on a finite group}   %   %   %   %   %   %   %   %   %   %   %   %   %   %   %
A finite group of order n provides a very simple example, although the integral is not what one usually thinks of as an integral. In this case, the function $\Phi$ is specified by a finite set of numbers, $\Phi(g1)$, $\Phi(g2)$, $...$, $\Phi(gn)$. It is natural to define the "integral" of $\Phi$ over the group to be the sum of the values that $\Phi$ assumes on the group. Thus we set
  \begin{equation}   %  =   =   =   =   =
  %\begin{split}
    \int \Phi(g) dg = \sum_{g \in G} \Phi(g) .
    \label{eq:5-1}
  %\end{split}
  \end{equation}

With this definition, the integral indeed has the property that 
  \begin{equation}   %  =   =   =   =   =
  %\begin{split}
    \int \Phi_{a}(g) dg = \int \Phi(ag) dg = \int \Phi(g) dg .
    \tag{{\theequation}a}
    \label{eq:5-2a}
  %\end{split}
  \end{equation}
This property is termed the \uline{left invariance} property of the integral. It is an immediate consequence of the fact that the sum over the group of $\Phi(ag)$ includes each value of $\Phi$ exactly once, since multiplication of the group with $a$ has only the effect of permuting the group
elements. It is worth mentioning that equation \eqref{eq:5-2a} depends on the fact that the summation is over the whole group; the result would certainly not be valid if the sum were over only a subset of the group.

It will be noted that the integral is also right invariant: the function $\Phi^{a}$ defined by $\Phi^{a} = \Phi\paren{ag}$ satisfies, for the same reasons that imply \eqref{eq:5-2a},
  \begin{equation*}   %  =   =   =   =   =
  %\begin{split}
    \int \Phi^{a}(g) dg = \int \Phi (ag) dg = \int \Phi(g) dg .
    \tag{{\theequation}b}
    \label{eq:5-2b}
  %\end{split}
  \end{equation*}
Moreover, for similar reasons,

linear and positive-definite; that is,
with the equality valid only if $\Phi$ is identically zero.

\section[Integration on a compact group]{Integration on a compact group. The Hurwitz integral}   %   %   %   %   %   %   %   %   %   %   %   %   %   %   %

The invariance of the integration over a finite group is rather trivial. We turn now to the case in which the group G is an n parameter Lie group.

\endinput
It is natural to attempt to define the integral as a definite integral over the group parameters. In doing this we meet immediately the problem that the various parameter domains may overlap, whereas the integral can hardly be expected to be invariant unless there is a 1-1 correspondence between group elements and coordinate points. To this end, we can delete from some of the parameter domains their intersection with others, so that the points of the resulting domains are in 1-1 correspondence with the group elements. As previously mentioned, the resulting domains may not be open, but this does not matter for defining the integral. Some of the domains may
be of lower dimension than n; the volume of such regions is zero. Actually, in all cases with which we shall be concerned, the original regions in parameter space can be chosen so that a single one, which will be described in all examples, covers the entire group except for a set of lower dimensionality. The integral can then be expressed
as the integral over one parameter domain, with the set of lower dimension ignored.
It is convenient in the discussion of invariant integration not to distinguish between points in the parameter space and group elements; this does not cause any mathematical difficulty if the assumptions of the previous paragraph are valid, and it leads to a simpler notation.
We attempt, therefore, to define the group integral in the form
tl>(g) dg =Jtl>(p)w(p) dp, (5.5)
the integral being over a single parameter domain. The function w(p) is a weight function that is included to give different weights to various regions of the parameter space. It can be seen that such a function is required, since if the group parameters are changed to
The left
The         assumed th if i::l.V is srr be denoted
We cons zero only i the volume of constant w(p) =w(ais therefor
tion (7) the
This argUl The rat bian. A p( The quotiE
Since i::l.V therefore:
It is obse tions vcip determin: tinuous, i
The ql The coor' p' �i::l.V' b
p' , the integral on the
Jtl>(p(p')) where
The notation op/Op' is
right hand side of (5) becomes
w(p(p'))    dp' =jtl>' (P') w, (P') dp' Jw'(p') = w(p(p'))     
used to denote the Jacobian o( p i , p2 , . . . , p n )
o(pIi,p'2, ���,p'n).
(5.6a)
(5.6b)
Equations (6) show that even if the weight function \Mere not present in one parametrization, it would be necessary to include it is another parametrization. The problem of constructing an invariant integral is, in fact, a problem of finding a suitable weight function.
The weight function can be multiplied by an arbitrary constant without affecting the invariance property. It is customary to "normalize" w either so that w(e) = 1, or so that the total group volume is 1. We will choofJe the former normalization.
It is possible to evaluate the weight function by considering the left invariance requirement for a particular function E which will be defined shortly. It is then possible to demonstrate that, because of the group associativity, the integral is left invariant.
Let us consider a volume element i::l.V in the neighborhood of the identity, or the origin in the parameter space. The function E is defined by E(p) = 1, p � i::l.V, and E(p) = 0 otherwise; E is sometimes called the characteristic function of i::l.V.
  TIONS
ower di-
not to
) ele-
  assump;impler
form (5.5)
     
.ghts to such a     to
(5.6a)
(5.6b)
resent another
tegral
ant nor-
)lume
the
NiH
:ause
,f the
is de-
INVARIANT INTEGRATION 53 The left invariance requirement is
jE(;P) w(p) dp = jE(ap) w(P) dp. (5.7)
The integrand on the left hand side is nonzero only in t::..V; if it is assumed that w(P) = w(0) = 1 in t::..V, and this is a good approximation if t::..V is small, the left hand integral is the volume of t::..V, which will be denoted by It::..VI �
We consider now the right hand integral. The integrand is nonzero only if ap �- t::..V, that is, only if p �- a-1 t::..V. We will denote t'he volume element a- 1 t::..V by t::..V'. We again make the assumption of constant w; this time over t::..V'. Since a- 1 �- t::..V', we c an take
w(p) =w(a-1). The right hand integral in (7), which is over only t::..V', is therefore w(a-1) It::..V' I where It::..v'.1 is the volume of t::..V'. Equa-
tion (7)
therefore -leads to
w(a-1)= It::..VI = lit::..V'1)-1 I t::..v'l \1t::.. v i
This argument is illustrated in Fig. 5-1.
The ratio of the two volume elements can be expressed as a Jaco-
bian. A point p �- t::..V is mapped into a point p' = a-lp =f(a-l ,p) �- t::..V'. The quotient of the volume elements is
It::..V'I =2.�'_I ofa -1 It::..vI op - aqf3 (a ,p)
I�
Since t::..V is located at e, this should be evaluated at p = O. We obtain
therefore,
replacing
a- l
by a,
w(a) =
l o f a
1- 1
I
(5.9)
oq (3 (a, 0)
It is observed that w is the inverse of the determinant of the functions va(3 defined by (3.8). It was shown in Section 3-4 that this determinant is nonzero. It follows that w(a) exists and, being continuous, is positive.
The quotient It::..VIIIt::..V' I can also be expressed as a Jacobian. The coordinates of a point p �- t::..V are given in terms of those of
p' �-t::..V' by p =f(a,p'). Therefore,
nes aI
It::..vl l o f
itN'I= oq(3(a,p') .
(5.8 )
    54
SPECIAL FUNCTIONS INVA
Fig; 5-1
If E(p) is nonzero only for p �, t::..V, E(ap) is nonzero only for p �, a- 1 t::..V. To ensure left invariance, the "volume" of a-1 t::..V must be reduced by the weight function to be that of t::..V.
This is to be evaluated at p = 0 or at p' = a-i. We obtain, therefore, again replacing a- i by a,
(5.10)
as an alternate expression for w.
It is possible to give a heuristic argument that the integral defined
with the weight function w is left invariant. Let $\Phi$ be a continuous function defined on G. We can approximate $\Phi$ arbitrarily closely by
a function that is constant over small regions of G. This amounts to writing $\Phi$ as a sum ci Ei + c2E2 + ... + cn En,where Ei is the charac-
teristic function of one of the small regions in which $\Phi$ is constant. Because of the linearity of the integral (5), it is sufficient to demonstrate the invariance of the integral for the functions Ei. Let t::..Vi
be th(
b be < W(
elem( chara wheth fore,
wher( ted w E(P) , the at
Camp integl the    of a-1 It is j the pl derly
Th
forme from and (J
THE e
for ar gral i and Pi
Proof of eac duct c
 INS
INVARIANT INTEGRATION
ore,
5.10)
efined us
y by
LtS to harac- ll1t. llon- Vi
55
We can express .6.Vi as b.6.V' where .6.VI = b- 1.6.Vi is a volume element containing e. We can write Ei(p) = E(b-1p) where E is the characteristic function of .6.VI, since E(b-1p) =1 or 0 depending on whether b-1p � .6.VI, that is, whether p � b.6.VI =.6.Vi . We can, therefore, write
j Ei(P) w(P) dp = fE(b-1p) w(p) dp = JE(p) w(P) dp (5.11)
where the second equality stems from the fact that we have constructed w(P) to ensure the left invariance of the integral of functions like E (P), namely characteristic functions of volume elements at e. On the other hand,
jEi(ap) w(P) dp = fE(b-1(ap� w(P) dp = JE�b-1a)p) w(P) dp = JE(p) w(P) dp.
Comparison with (11) demonstrates the desired left invariance of the integral of Ei' The first step in the above calculation stems from
the fact that Ei(ap) and E(b-1(ap� are each characteristic functions of a-1.6.Vi since each of these is 1 for p � a-1.6.Vi and 0 elsewhere.
It is important to observe that the associative law is necessary in the proof in equating E(b-1(ap� and E�b-1a)p). The ngeometryfl underlying the argument is illustrated in Fig. 5-2.
These rather intuitive arguments will now be converted into a formal theorem that may also satisfy those who doubt the transition from the quotient of volume elements to Jacobians of equations (9) and (10).
THEOREM 5-1. If w(p) is defined by equation (9)
Jet(P) w(P) dp =jCI>(ap) w(p) dp (5.12)
for any continuous function CI> and any group element a, if the integral is taken over the whole group. The integral so defined is linear and positive-definite.
Proof. We again consider equation (3 .9). If we take the determinant of each side of (3 .9) and use the result that the determinant of the product of two matrices is the product of their determinants, we obtain
af O! w(f(p,q�-1=Iaqf3 (p,q) Iw(q)-1
be the volume element whose characteristic function is Ei and let b be an element of .6.Vi'
     56
SPECIAL FUNCTIONS
b.V� =bb.V' 1
a -1b.Vi = a -1(bb. V')
= (a -1b)b.V' acco
Fig. 5-2
If Ei(p) is nonzero only in b.Vi, Ei(ap) is nonzero only in a-1b.Vi' The "group" volume of these two regions is
the same, each being the volume of b.VI .
or, replacing p by a-1,
chan chan grou f(a- 1 that pend
T
sinc
defil
wei
w(f(.
1
any
that
is c
sine
spa<
coni
j
grot
0:'
v f3 The
The not grOt inte
exiE
5-3,
It is at this point that the associative law is used, since (3.9) was derived from it.
We now change the variables of integration in the right-hand side of (12) to q = ap = f(a,p); then p = a-1q =f(a-1, q). The right-hand side of (12) becomes, including the Jacobian op;8q,
0:'
of (-1 oqf3 a
, q )
I dq = j\Phi(q) w(q) dq, can riec
reS1
(5.13)
INV
 INS
1')
VI
INVARIANT INTEGRATION
57
).13)
de-
:ide
I
according to (13). This is, however, the left-hand side of (12). In changing variables in the multiple integral, it is necessary also to change the integration domain. Since the integral is over the whole group, the domain is not essentially changed when p is replaced by f(a- 1, p) but is merely rearranged. It can be concluded, therefore, that (12) is valid. It is important to note, though, that this again depends on the integral's being extended over the whole group.
The integral defined is obviously linear in the function cI> and, since we have argued that w(P) > 0, the integral satisfies the positive definiteness condition of equation (4). If we put q = a in equation (13),
we find equation (10) as a suitable alternate expression for w(a), since w(f(a-1, a� = w(e) = 1.
The above calculation of w and the ensuing argument is valid for any Lie group. The integral (5) may not always exist, however, so that the argument applies only if the integral is finite. If the group is compact, however, the integral must exist for all continuous cI> since the integration is over a bounded region in the parameter space, and the funotions cI> and w must be bounded since they are continuous on a compact set.
As an example, the weight function for the non-Abelian affine group considered in Section 4-3 will be calculated. The functions vO'.(3 for this       can be calculated from equations (3.8) and (4.15). They are
1 v11 = 1 V 12 = 0 V 21 = 0 V 22 = eP .
The invariant weight function is therefore e-P 1. The affine group is not compact, so that even for continuous cI>, the integral over the group may not converge. If, however, a function cI> is such that the integral
JOO 1 f 00
-00 e-p dp1 _00 dp2 cI>(p1,p2)
exists, it is left invariant.
�
6-3. RIGHT INVARIANT INTEGRATION
It is clear that aright invariant integral over a compact Lie group can also be defined. If the procedure of the previous section is carried through, one obtains a right invariant weight function wr; the result analogous to equation (9) is
  58
SPECIAL FUNCTIONS
INV
and the result analogous to equation (10) is ofa
wr(a) ::::
-1
1� q2 1
1   w(a).
wr(a) ::::
(5.14)
(5.15)
Of a
Iorft (0, a)
1-
1
THE
Pro( by q of th abov
whe
and
I
weI
a(bJ:
wr(a) :::: 1 op{3 (a, a-i) 1 .
In general, the weight function wr for right invariant integration may differ from the weight function w. For example, in the affine       considered previously,
It is obvious that the left and right invariant weight functions must be the same in any Abelian group. It is also true, but far from obvi-
The This result will now be proved. We will first sketch the idea of the qa(C
ous, that the two weight functions are equal in any compact Lie group.
proof and then give a formal theorem. Jaci We suppose first that for some group element a, wr(a) :::: Aw(a) M(a
where A   1. It has been seen that if b..V is a volume element con- taining e, Ib..VI :::: w(a) Ia.6.vi. Similarly, for any .6.U we have Ib..U I :::: wr(a) I.6.Ual. This is true in particular for.6.U :::: a .6.Va- 1,so that
la .6.V a-11 :::: wr (a) Ia.6.VI :::: Aw(a) Ia.6.vi :::: AI.6.VI. It follows that transforming any volume element b..V to a .6.V a-1 changes its volume
by a factor A. The set a .6.V a-1 is again a volume element containing e, and hence la(a b..V a- 1)a- 1 1 :::: A2 1.6.VI. It follows that transforming .6.V to a2 .6.V a-2 changes its volume by a factor A2. It can easily be proved inductively that transforming .6.V to an b..V a-n transforms
its volume by a factor An.
The set of elements an, n:::: 1, 2 , ... is infinite and, since the
that
Lie lim
   
thE ho' im M(
grl
group is compact, must have at least one limit element 0 � G. The
transformation of b..V to 0 .6.V 0-1 must change the volume of b..V by
a factor that is either zero or infinite, because of the continuity of
the group operations. On the other hand, the transformation a' :::: Th; o a 0-1 is an invertible mapping of the group onto itself, so that the val corresponding coordinate transformation must have a finite Jacobian. the The resolution of this contradiction is that A is necessarily 1 and
w(a) :::: wr(a). We will now attempt to put these considerations into a precise form.
  NS
.14)
.15)
nay lp
INVARIANT INTEGRATION
THEOREM 5-2. If G is a compact Lie group, wr(a) = w(a) .
59
lUst 'vi- 'oup.
vI
at
it lme rring ning be
le )y
e
ian. ) a
where we have made use of (9) and (15). It is observed that M(a) = A., and we therefore wish to show that M(a) = l.
It will first be shown that M(a) M(b) = Meab). For this purpose, we consider the composition transformation qa ('4>(P� = qa(bpb- 1) = a(bpb- 1)a-i. This is the same as (ab)p(ab)-1 = qab(P) , so that
qa (Qb (P� = qab(p), (5.17)
The value of the Jacobian of qab at e is M(ab). The Jacobian of qa(Qb(P� is the product of the Jacobians of qa and '4>; at p = e the Jacobian of   is M(b) and, since '4>(e) = e, the Jacobian of qa is M(a). We can conclude that M(a) M(b) = M(ab). From this it follows thatM(an) = [M(a)]n.
The above considerations apply in any Lie group. In a compact Lie group, however, the sequence of elements a, a2, a3, ��� has a limit element c � G; that is, there is a subsequence ani, an2, ...
that converges to c. In view of the continuity of the group operations, the Jacobians M(al1j) = [M(a)]nj must converge to M(c), which must, however be a finite number (M(c) ,c 0 and M(c) ,c 00) since Clc is an invertii)le transforraatiull. If M(a)nj converges to a finite number, M(a) is necessarily 1, showing thatw(a) = wr(a).
It can be shown as a corollary to Theorem 5-2 that in a compact group
Proof. We consider, for each a � G, a transformation qa (p) defined by qa (P) = apa- 1 = f(f(a,p), a-i). It will be shown that the value at e
of the Jacobian of this transformation is the quotient A. discussed above. The value of the Jacobian at e will be denoted by M(a). Then
Iafa - a{Y I M(a)= ap'Y(f(a,p),a1) aqJ3(a,p) p=O
af a -1llaf'Y \- wr(a) =lap'Y (a,a ) aqi3(a,o) - w(a) ,
(5.16)
f<I>(q-1) w(q)dq =      w(q) dq.
(5.18)
That is, the invariant integral over a compact group also has the invariance property of equation (2c). To show this we put p = q-1 in the left-hand side of (18) to obtain
J<I>(q-1) w(q) dq =
f<I>(p) w(q)    (p)1 dp,
(5.19)
 60
SPEeIAL FUNCTIONS
where the integral is again the whole group. It is necessary to take the absolute value of the Jacobian, since it will be found that this particular Jacobian is negative. We now recall equation (3.7) which can be written in the present notation as
aq' Y _ a f ' Y a f a _ 1
apf3 (p) - - apa (0, q) aqf3 (p ,p). (5.20)
Taking the Jacobian of each side of (20), and using (10) and (14), we find
aq()=_w(p) =_w(p) ap p wr(q) w(q) ,
provided the group is compact. Substituting this result into (19) shows that
f!l>(q-1) w(q) dq = !l>(P) w(p) dp.
   
sen deb
is t to y lati in t
gro infi in t is 1 thr pOI
Th( tiOl hoc
cal mu pal    
by an( thE sic
\endinput